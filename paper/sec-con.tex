\section{Conclusion \label{sec-con}}
In this paper, we proposed a novel heterogeneous information network embedding based approach (\ie HERec) to effectively utilizing auxiliary information in HINs for recommendation. We designed a new random walk strategy based on meta-paths to derive more meaningful node sequences for network embedding. Since embeddings based on different meta-paths contain different semantic, the learned embeddings were further integrated into an extended matrix factorization model using a set of fusion functions. Finally, the extended matrix factorization model together with fusion functions were jointly optimized for the rating prediction task. HERec aimed to learn useful information representations from HINs guided by the specific recommendation task, which distinguished the proposed approach from existing HIN based recommendation methods. Extensive experiments on three real datasets demonstrated the effectiveness of HERec. We also verified the ability of HERec to alleviate cold-start problem and examine the impact of meta-paths on performance.

As future work, we will investigate into how to apply deep learning methods (\eg convolutional neural networks, auto encoder)
to better fuse the embeddings of multiple meta-paths. In addition, we only use the meta-paths which have the same starting and ending types to effectively extract network structure features in this work. Therefore, it is interesting and natural to extend the proposed model to learn the embeddings of any nodes with arbitrary meta-paths. As a major issue of recommender systems, we will also consider how to enhance the explainablity of the recommendation method based on the semantics of meta-paths.
%As part of future work, we are interested in applying deep learning methods (\emph{e.g.} convolutional neural networks and auto encoders) to better fuse embeddings of various meta-paths. In addition, the proposed HIN embedding can also be employed for other applications.
